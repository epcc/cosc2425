\section{Control Flow}\label{control-flow}

Control flow is the order in which individual instructions are executed.
In the MSP430 microcontroller and most architectures, instructions are
executed in the order they appear in the program. The Program Counter
(PC) register keeps track of the address of the next instruction to be
executed. Before a program can be executed, the PC needs to be set to
the address of the first instruction of the program. For example, take
the following program:

\begin{verbatim}
A000: MOV r4, r5
A002: ADD r10, r5
\end{verbatim}

The first instruction lives in address \textbf{A000}. The instruction
occupies 1 word (2 bytes), so the next instruction lives in address
A002.

Before the program can be executed, the Program Counter (PC) register
will be set to \textbf{A000}. The processor will execute the instruction
at addres \textbf{A000}, then it will increment the PC so that it points
to the next instruction. In this example, the PC will be now set to
\textbf{A002}. \emph{Branching} is changing the flow when certain
conditions are true.

Remember that on the MSP430 the size of the instruction depends on the
addressing mode of the parameters.

\subsection{Exercise}\label{exercise}

Given a small sequence of instructions and a starting address, figure
out the address of every instruction.

\section{Branching}\label{branching}

\subsection{Conditions and Loops}\label{conditions-and-loops}

All computer programs need to make decisions. The familiar
\texttt{if-then-else} construct found in Java-like programming languages
tests a condition and then executes a block of code if the condition is
true or a different block if the condition is false. Such constructs can
test for one or more conditions. For example, to check whether the
variable \texttt{year} is a leap year, one might write:

\begin{Shaded}
\begin{Highlighting}[]
\KeywordTok{if}\NormalTok{(year % }\DecValTok{4} \NormalTok{== }\DecValTok{0} \NormalTok{&& year % }\DecValTok{100} \NormalTok{!= }\DecValTok{0}\NormalTok{) || (year % }\DecValTok{400} \NormalTok{== }\DecValTok{0}\NormalTok{))}
\NormalTok{\{}
     \NormalTok{isLeapYear = }\KeywordTok{true}\NormalTok{;}
\NormalTok{\}}
\KeywordTok{else}
\NormalTok{\{}
     \NormalTok{isLeapYear = }\KeywordTok{false}\NormalTok{;}
\NormalTok{\}}
\end{Highlighting}
\end{Shaded}

Such comparisons are ultimately carried out by the processor in a simple
way: two values are compared by performing a subtraction. A subtraction
---just like most arithmetic operations--- sets \emph{arithmetic flags}
to record whether the result is correct, has the wrong sign, or is
incomplete. Programmers, in turn, can use these flags to determine the
relationship between two values. Special instructions called \emph{jump
instructions} are then used to alter the flow of the program if certain
flags are set.

When several conditions are to be tested, each one has to be checked by
an individual subtraction operation as there is no way of checking
multiple conditions at once.

\subsection{Comparisons in a Computer}\label{comparisons-in-a-computer}

Comparing $A$ and $B$ is simple: we know that if $A=B$, then $A-B=0$. If
$A>B$, then $A-B>0$; and if $A < B$, then $A-B<0$; however, we cannot
blindly trust the result of $A-B$ in a computer: because numbers are
stored in a finite amount of bits, arithmetic operations may produce
results too big (or too small) to be represented. We need to handle such
situations adequately.

\subsection{The Negative (N) and Zero (Z)
Flags}\label{the-negative-n-and-zero-z-flags}

The negative and zero flags are set when the result is negative (when
the most significant bit is $1$) and when the result is zero. The carry
and overflow flags have subtle implications that merit more detail.

\subsection{The Carry (C) Flag}\label{the-carry-c-flag}

The carry flag is set when a carry happens in the most significant bit
of an arithmetic operation. For example, adding the 16-bit numbers
\texttt{0x0FFF} and \texttt{0xFF00} results in a carry in the most
significant bit:

\begin{verbatim}
 11111111        
  0000111111111111
+ 1111111100000000
------------------
  0000111011111111
\end{verbatim}

The carry flag signals that the result does not fit in the number of
bits we are working with. In the previous example, the carry is set
because the actual result does not fit in 16 bits. However, if we had 17
bits to store the result instead of 16, the carry of the $16^{th}$ bit
would fall naturally into the $17^{th}$ bit and there would be no actual
carry. Obviously, the number of bits in a word is always limited and a
carry is always possible.

As mentioned before, we perform subtractions to compare two values.
However, subtractions in most processors are performed not by
subtracting bit-by-bit, but by adding one operand to the \textbf{2's
complement} of the other operand. So, to compute $5-3$ the MSP430 adds
$5$ to the \textbf{2's complement} of 3 (which is $-3$) as shown below:

\begin{verbatim}
 11111111111111 1 
  0000000000000101 (the number 5)
+ 1111111111111101 (the 2's complement of 3, -3)
------------------
  0000000000000010 (2, the correct result of 5-3)
\end{verbatim}

However, notice how the carry flag is not set when the second operand is
greater than the first, like when subtracting $5-6$:

\begin{verbatim}
  0000000000000101 (the number 5)
+ 1111111111111010 (the 2's complement of 6, -6)
------------------
  1111111111111111 (-1, the correct result) 
\end{verbatim}

Knowing this, we can compare two unsigned numbers $A$ and $B$ by
subtracting $A-B$ and checking the carry flag. From the above examples,
we can deduce that $A \geq B$ if the carry flag is set, and that $A < B$
if it is not set. The opposite conditions, namely $A \leq B$ and
$A > B$, can be checked by reversing the subtraction order.

\subsection{Signed Numbers}\label{signed-numbers}

This approach, however, does not always work with signed numbers. For
small negative values it works with no problems; for example, if $A=-5$
and $B=-3$, everything goes as expected when performing $(-)5 - (-)3$:

\begin{verbatim}
                    11  (no carry)
       1111111111111011 (-5)
     + 0000000000000011 (the 2's complement of -3, that is, 3)
     ------------------
       1111111111111110 (-2) 
\end{verbatim}

Performing $A-B$ does not produce a carry and we can deduce that
$A < B$, which is true because $-5 < -3$. If we reverse the subtraction
order, everything still works as expected: the carry flag is set and we
can assume that $-3 \geq -5$, which is true.

This does not work when the signed numbers get close to their ranges.
For example, if we compare the smallest possible 16-bit negative value,
$-32,768$ to $2$ by performing a subtraction, the carry flag does not
tell us the whole story:

\begin{verbatim}
      1
       1000000000000000 (-32,768, the smallest negative value)
     + 1111111111111110 (the 2's complement of 2, that is, -2)
     ------------------
       0111111111111110 (32,766) 
\end{verbatim}

Here, the carry flag was set, telling us incorrectly that
$-32,768 \geq 2$. However, the exact same operation makes complete sense
if we interpret the operands as unsigned numbers:

\begin{verbatim}
      1
       1000000000000000 (32,768)
     + 1111111111111110 (the 2's complement of 2, that is, -2)
     ------------------
       0111111111111110 (32,766) 
\end{verbatim}

The result is the same, but now looking at the operands as positive
values, the carry bit is telling us that $32,768 \geq 2$, which is
correct. It is clear that we will need to use another approach when
dealing with signed numbers.
